\documentclass[a4paper,12pt]{article}
\usepackage{geometry}
\usepackage{longtable}
\usepackage{booktabs}
\usepackage{graphicx}

\usepackage{array}
\usepackage{longtable}
\usepackage{booktabs}

% 在导言区添加(只需一次)
\renewcommand{\arraystretch}{1.2} % 行距略大,增强可读性
\newcolumntype{L}[1]{>{\raggedright\arraybackslash}p{#1}} % 自定义左对齐列宽

\geometry{left=2.5cm,right=2.5cm,top=2.5cm,bottom=2.5cm}

\begin{document}

\begin{center}
    \Large \textbf{Verification Plan for RISC-V ALU and Execute Stage}\\[6pt]
    \normalsize Shengjie Chen
\end{center}

\section{Verification Scope and Objectives}

This verification plan targets two cascaded modules in the RISC-V pipeline:
\begin{itemize}
    \item \textbf{ALU (Arithmetic Logic Unit):} Performs arithmetic, logic, shift, and comparison operations based on the 4-bit control signal.
    \item \textbf{Execute Stage:} Integrates the ALU, operand forwarding logic, and branch handling. It computes results and forwarding data for subsequent pipeline stages.
\end{itemize}
\textbf{Objectives:}
\begin{enumerate}
    \item Verify functional correctness of ALU operations and flag generation (zero, overflow).
    \item Verify operand selection and forwarding paths within the execute stage.
    \item Ensure correct generation of control signals (\texttt{pc\_src}, \texttt{jalr\_flag}, etc.).
    \item Achieve functional and code coverage closure with UVM-based verification.
\end{enumerate}

\section{Design Interface Summary}

% ======== ALU TABLE ========
\subsection*{ALU}
\begin{longtable}{@{}L{3.5cm}L{1.8cm}L{8.5cm}@{}}
\caption{Interface Summary of the ALU}\\
\toprule
\textbf{Signal} & \textbf{Direction} & \textbf{Description} \\ 
\midrule
\endfirsthead

\toprule
\textbf{Signal} & \textbf{Direction} & \textbf{Description} \\ 
\midrule
\endhead

\midrule
\multicolumn{3}{r}{\textit{(continued on next page)}} \\
\bottomrule
\endfoot

\bottomrule
\endlastfoot

\texttt{control[3:0]} & Input & Operation select code controlling the arithmetic, logical, shift, or comparison behavior of the ALU. \\[2pt]
\texttt{left\_operand[31:0]} & Input & Left-hand operand used as the first input to the operation. \\[2pt]
\texttt{right\_operand[31:0]} & Input & Right-hand operand or shift amount, depending on operation type. \\[2pt]
\texttt{result[31:0]} & Output & Computed result of the selected ALU operation. \\[2pt]
\texttt{zero\_flag} & Output & Asserted high when \texttt{result} equals zero; used for branch decision logic. \\[2pt]
\texttt{overflow} & Output & Indicates a signed arithmetic overflow during \texttt{ADD} or \texttt{SUB} operations. \\

\end{longtable}


\subsection*{Execute Stage}
\begin{longtable}{@{}L{3.5cm}L{1.8cm}L{8.5cm}@{}}
\caption{Interface Summary of the Execute Stage}\\
\toprule
\textbf{Signal} & \textbf{Direction} & \textbf{Description} \\ 
\midrule
\endfirsthead

\toprule
\textbf{Signal} & \textbf{Direction} & \textbf{Description} \\ 
\midrule
\endhead

\midrule
\multicolumn{3}{r}{\textit{(continued on next page)}} \\
\bottomrule
\endfoot

\bottomrule
\endlastfoot

\texttt{data1}, \texttt{data2} & Input & Source register operands from the decode stage. \\
\texttt{immediate\_data} & Input & Immediate value decoded from instruction. \\
\texttt{pc\_in} & Input & Current program counter value entering the execute stage. \\
\texttt{control\_in} & Input & Control structure containing ALU operation type, instruction encoding, and pipeline control signals. \\
\texttt{wb\_forward\_data} & Input & Data forwarded from the write-back stage for hazard resolution. \\
\texttt{mem\_forward\_data} & Input & Data forwarded from the memory stage for hazard resolution. \\
\texttt{forward\_rs1}, \texttt{forward\_rs2} & Input & Multiplexer selectors determining which data source to forward for operands 1 and 2. \\
\texttt{control\_out} & Output & Propagated control signals to the next pipeline stage. \\
\texttt{alu\_data} & Output & Final ALU computation result. \\
\texttt{memory\_data} & Output & Data prepared for memory write operations (valid for S-type instructions). \\
\texttt{pc\_src} & Output & Branch decision flag indicating whether the next PC should be redirected. \\
\texttt{jalr\_target\_offset} & Output & Calculated target address offset for JALR instructions. \\
\texttt{jalr\_flag} & Output & Indicates that a JALR-type instruction is being executed. \\
\texttt{pc\_out} & Output & Program counter value passed to the next stage. \\
\texttt{overflow} & Output & Overflow indicator propagated from the ALU operation. \\

\end{longtable}


\section{Verification Environment Overview}

\subsection{Testbench Architecture}
A UVM-based testbench will be built using Easier UVM generator, including:
\begin{itemize}
    \item \textbf{Interface:} Connects testbench to DUT (ALU or execute\_stage).
    \item \textbf{Driver:} Drives randomized transaction stimuli.
    \item \textbf{Monitor:} Captures DUT outputs for checking and coverage.
    \item \textbf{Scoreboard:} Compares DUT outputs with reference model.
    \item \textbf{sequence item:} The data generated from the randomization.
    \item \textbf{Sequence/Sequencer:} Generates constrained random operations and operand patterns.
    \item \textbf{Coverage Collector:} Records coverage metrics.
\end{itemize}

% \begin{center}
%     \includegraphics[width=0.8\textwidth]{uvm_env.png}
% \end{center}

\section{Functional Verification Plan}

\subsection{ALU Verification Items}

\textbf{Operation Coverage:}
\begin{longtable}{@{}lll@{}}
\toprule
\textbf{Category} & \textbf{Operation} & \textbf{Verification Goal} \\ \midrule
Logic & AND, OR, XOR & Verify bitwise correctness. \\
Arithmetic & ADD, SUB & Verify signed arithmetic and overflow. \\
Comparison & SLT, SLTU & Verify signed/unsigned comparison correctness. \\
Shift & SLL, SRL, SRA & Verify correct shift amount and direction. \\
Immediate & LUI & Verify correct upper-immediate load. \\
Branch & Bxx (BEQ, BNE, etc.) & Verify zero-flag generation correctness. \\ \bottomrule
\end{longtable}
\textbf{Flag Verification:}
\begin{itemize}
    \item \texttt{zero\_flag} = 1 iff result == 0.
    \item \texttt{overflow} = 1 iff ADD/SUB crosses signed boundary.
\end{itemize}

\subsection{Execute Stage Verification Items}

\textbf{Forwarding Logic:}
\begin{itemize}
    \item Verify correct selection from \texttt{data1}, \texttt{data2}, \texttt{mem\_forward\_data}, and \texttt{wb\_forward\_data}.
    \item Check data hazards resolution in cases FORWARD\_FROM\_EX and FORWARD\_FROM\_MEM.
\end{itemize}
\textbf{Control and Branch:}
\begin{itemize}
    \item Verify \texttt{pc\_src} asserted only for valid branch.
    \item Verify \texttt{jalr\_flag} and \texttt{jalr\_target\_offset} for jump instructions.
\end{itemize}
\textbf{ALU Integration:}
\begin{itemize}
    \item Cross-check ALU output consistency with Execute Stage’s \texttt{alu\_data}.
    \item Verify propagation of overflow and zero\_flag signals.
\end{itemize}

\section{Bug Detection Strategy}

\begin{itemize}
    \item Use \textbf{scoreboard} to compare DUT outputs with a SystemVerilog reference model.
    \item Add \textbf{assertions (SVA)}:
    \begin{itemize}
        \item \texttt{assert property (result == 0 -> zero\_flag == 1);}
        \item \texttt{assert property (control inside \{ADD,SUB\} |-> overflow == expected);}
        \item \texttt{assert property (forwarding path matches selected signal);}
    \end{itemize}
    \item Use directed tests for corner cases (e.g., overflow, max/min shift).
\end{itemize}

\section{Coverage Plan}

\subsection{Code Coverage}
Collected automatically by simulator:
\begin{itemize}
    \item Statement coverage
    \item Branch coverage
    \item Toggle coverage
\end{itemize}

\subsection{Functional Coverage}

\textbf{A covergroup will be defined on the transaction level to track ALU operation types and key functional signals:}

\begin{verbatim}
// Covergroup definition for ALU operations and overflow conditions
covergroup alu_cg @(posedge vif.clk);

  // Control operation type classification
  coverpoint tr.control {
    // Logical operations
    bins logic_ops[]  = {ALU_AND, ALU_OR, ALU_XOR};

    // Arithmetic operations with possible overflow
    bins arith_ops[]  = {ALU_ADD, ALU_SUB};

    // Shift operations
    bins shift_ops[]  = {ALU_SLL, ALU_SRL, ALU_SRA};

    // Comparison operations
    bins cmp_ops[]    = {ALU_SLT, ALU_SLTU};

    // Illegal or default operation (unexpected control)
    bins illegal_ops  = default;
  }

  // Capture overflow condition
  coverpoint tr.overflow;

  // Ensure overflow only occurs in ADD or SUB operations
  cross control, overflow;

endgroup
\end{verbatim}
\textbf{Execute Stage coverage will cross:}
\begin{itemize}
    \item Operand forwarding paths $\times$ control type.  
    \textit{→ Ensures all instruction types are tested under every forwarding condition.}

    \item Branch decisions ($pc\_src$) $\times$ zero\_flag.  
    \textit{→ Confirms branch logic behaves correctly for both taken and not-taken cases.}

    \item Instruction encoding types ($encoding\_t$).  
    \textit{→ Verifies that all instruction formats (R/I/S/B/U/J) are executed in simulation.}

    \item Immediate data extension cases.  
    \textit{→ Ensures both sign-extension and zero-extension immediates are covered.}

    \item Control signal forwarding ($control\_in$ $\times$ $control\_out$).  
    \textit{→ Confirms control signals are properly propagated between pipeline stages.}

    \item JALR branch target computation ($jalr\_flag$ $\times$ $jalr\_target\_offset$).  
    \textit{→ Validates that JALR-type branch targets are calculated correctly.}
\end{itemize}


\section{Verification Closure Criteria}

\begin{itemize}
    \item All functional coverage points reached ($\geq 95\%$).
    \item Code coverage $\geq 90\%$.
    \item All assertions pass without failure.
    \item No mismatches reported by scoreboard.
    \item All directed corner cases verified.
\end{itemize}

\section{Deliverables}

\begin{itemize}
    \item UVM environment source files.
    \item Verification plan and coverage report.
    \item Simulation waveforms for representative cases.
    \item Summary report of detected bugs and fixes.
\end{itemize}

\end{document}
