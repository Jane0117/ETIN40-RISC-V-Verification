\documentclass[a4paper,12pt]{article}
\usepackage{geometry}
\usepackage{longtable}
\usepackage{booktabs}
\usepackage{graphicx}

\usepackage{array}
\usepackage{longtable}
\usepackage{booktabs}

% 在导言区添加(只需一次)
\renewcommand{\arraystretch}{1.2} % 行距略大,增强可读性
\newcolumntype{L}[1]{>{\raggedright\arraybackslash}p{#1}} % 自定义左对齐列宽

\geometry{left=2.5cm,right=2.5cm,top=2.5cm,bottom=2.5cm}

\begin{document}

\begin{center}
    \Large \textbf{Verification Plan for RISC-V ALU and Execute Stage}\\[6pt]
    \normalsize Shengjie Chen; Peihao Sun; Pengxin Luan
\end{center}

\section{Verification Scope and Objectives}

This verification plan targets two cascaded modules in the RISC-V pipeline:
\begin{itemize}
    \item \textbf{Decode Stage} control read and write from register filse and judge if the input instruction is invalid.
    \item \textbf{Execute Stage:} Integrates the ALU, operand forwarding logic, and branch handling. It computes results and forwarding data for subsequent pipeline stages.
\end{itemize}
\textbf{Objectives:}
\begin{enumerate}
    \item Verify read and write of the register files could work as expected.
    \item Verify if the decode stage could generate illegal instruction signal correctly.
    \item Verify operand selection and forwarding paths within the execute stage.
    \item Ensure correct generation of control signals (\texttt{pc\_src}, \texttt{jalr\_flag}, etc.).
    \item Achieve functional and code coverage closure with UVM-based verification.
\end{enumerate}

\section{Design Interface Summary}

% ======== DECODE STAGE TABLE ========
\subsection*{Decode Stage}
\begin{longtable}{@{}L{3.5cm}L{1.8cm}L{8.5cm}@{}}
\caption{Interface Summary of the Decode Stage}\\
\toprule
\textbf{Signal} & \textbf{Direction} & \textbf{Description} \\ 
\midrule
\endfirsthead

\toprule
\textbf{Signal} & \textbf{Direction} & \textbf{Description} \\ 
\midrule
\endhead

\midrule
\multicolumn{3}{r}{\textit{(continued on next page)}} \\
\bottomrule
\endfoot

\bottomrule
\endlastfoot

% --- Input Signals ---
\texttt{clk} & Input & Main clock signal driving synchronous logic (e.g., register file). \\[2pt]
\texttt{reset\_n} & Input & Active-low reset signal. \\[2pt]
\texttt{instruction} & Input & The 32-bit instruction (\texttt{instruction\_type}) from the fetch stage. \\[2pt]
\texttt{pc\_in[31:0]} & Input & The Program Counter (PC) address of the current instruction. \\[2pt]
\texttt{write\_en} & Input & Register write-enable signal, typically from the Writeback (WB) stage. \\[2pt]
\texttt{write\_id[4:0]} & Input & Target register ID for the writeback operation. \\[2pt]
\texttt{write\_data[31:0]} & Input & Data to be written into the register file from the Writeback stage. \\[2pt]

% --- Output Signals ---
\texttt{reg\_rd\_id[4:0]} & Output & Destination register ID (rd) decoded from the instruction. \\[2pt]
\texttt{read\_data1[31:0]} & Output & Data for source operand 1 (rs1), read from the register file. \\[2pt]
\texttt{read\_data2[31:0]} & Output & Data for source operand 2 (rs2), read from the register file. \\[2pt]
\texttt{immediate\_data[31:0]} & Output & Sign-extended immediate value decoded from the instruction. \\[2pt]
\texttt{pc\_out[31:0]} & Output & Program Counter (PC) value passed to the next stage (same as \texttt{pc\_in}). \\[2pt]
\texttt{instruction\_illegal} & Output & Flag asserted if the instruction is unrecognized or registers are invalid. \\[2pt]
\texttt{control\_signals} & Output & Generated control signals (\texttt{control\_type}) for subsequent pipeline stages. \\

\end{longtable}


\subsection*{Execute Stage}
\begin{longtable}{@{}L{3.5cm}L{1.8cm}L{8.5cm}@{}}
\caption{Interface Summary of the Execute Stage}\\
\toprule
\textbf{Signal} & \textbf{Direction} & \textbf{Description} \\ 
\midrule
\endfirsthead

\toprule
\textbf{Signal} & \textbf{Direction} & \textbf{Description} \\ 
\midrule
\endhead

\midrule
\multicolumn{3}{r}{\textit{(continued on next page)}} \\
\bottomrule
\endfoot

\bottomrule
\endlastfoot

\texttt{data1}, \texttt{data2} & Input & Source register operands from the decode stage. \\
\texttt{immediate\_data} & Input & Immediate value decoded from instruction. \\
\texttt{pc\_in} & Input & Current program counter value entering the execute stage. \\
\texttt{control\_in} & Input & Control structure containing ALU operation type, instruction encoding, and pipeline control signals. \\
\texttt{wb\_forward\_data} & Input & Data forwarded from the write-back stage for hazard resolution. \\
\texttt{mem\_forward\_data} & Input & Data forwarded from the memory stage for hazard resolution. \\
\texttt{forward\_rs1}, \texttt{forward\_rs2} & Input & Multiplexer selectors determining which data source to forward for operands 1 and 2. \\
\texttt{control\_out} & Output & Propagated control signals to the next pipeline stage. \\
\texttt{alu\_data} & Output & Final ALU computation result. \\
\texttt{memory\_data} & Output & Data prepared for memory write operations (valid for S-type instructions). \\
\texttt{pc\_src} & Output & Branch decision flag indicating whether the next PC should be redirected. \\
\texttt{jalr\_target\_offset} & Output & Calculated target address offset for JALR instructions. \\
\texttt{jalr\_flag} & Output & Indicates that a JALR-type instruction is being executed. \\
\texttt{pc\_out} & Output & Program counter value passed to the next stage. \\
\texttt{overflow} & Output & Overflow indicator propagated from the ALU operation. \\

\end{longtable}


\section{Verification Environment Overview}

\subsection{Testbench Architecture}
A UVM-based joint testbench will be built using Easier UVM generator, including:
\begin{itemize}
    \item \textbf{Interface:} Connects the joint testbench to the cascaded DUT (decode\_stage + execute\_stage).
    \item \textbf{Driver:} Drives randomized transaction stimuli.
    \item \textbf{Monitor:} Captures DUT outputs for checking and coverage.
    \item \textbf{Scoreboard:} Compares DUT outputs with reference model.
    \item \textbf{sequence item:} The data generated from the randomization.
    \item \textbf{Sequence/Sequencer:} Generates constrained random operations and operand patterns.
    \item \textbf{Coverage Collector:} Records coverage metrics.
\end{itemize}

The joint testbench for decode+execute cascade should look as follows:
\begin{figure}[h]
    \centering % 3. 将图片居中
    \includegraphics[width=0.8\textwidth]{bd_new.png} % 4. 插入图片
    \caption{joint decode+execute verification structure} % 5. 为图片添加标题
    \label{fig:joint_tb_diagram} % 6. 添加标签,用于正文引用
\end{figure}

\begin{itemize}
    \item \textbf{Interfaces/Agents:} Active agents drive \texttt{decode\_in}, \texttt{decode\_wb}, and \texttt{execute\_in}. Passive agents monitor \texttt{decode\_out} and \texttt{execute\_out} from the cascaded DUT.
    \item \textbf{Sequences:} \texttt{decode\_in\_default\_seq} and \texttt{decode\_wb\_default\_seq} generate instruction streams and writeback updates; \texttt{execute\_in\_default\_seq} drives execute inputs from \texttt{exe\_v1}.
    \item \textbf{Scoreboard/Reference:} \texttt{joint\_scoreboard} checks decode legality, control/pc consistency, regfile read behavior, and aligns decode-to-execute results using \texttt{seq\_id} with expected outputs from \texttt{joint\_ref\_model}.
\end{itemize}
% \begin{center}
%     \includegraphics[width=0.8\textwidth]{uvm_env.png}
% \end{center}

\section{Functional Verification Plan}

\subsection{Decode stage Verification Items}

\begin{itemize}
    \item \textbf{Illegal Opcode:} Verify that when the \texttt{instruction} input is an invalid or unsupported combination of \texttt{opcode}, \texttt{funct3}, or \texttt{funct7}, \texttt{decode\_failed} is asserted, subsequently asserting \texttt{instruction\_illegal}.

    \item Verify \texttt{immediate\_data} is correctly sign-extended and formatted for all five immediate types (I, S, B, U, J) based on the \texttt{controls.encoding} signal.

    \item \textbf{Read Operation:} Verify \texttt{read\_data1} and \texttt{read\_data2} correctly output the values from the register locations specified by \texttt{instruction.rs1} and \texttt{instruction.rs2}.

    \item \textbf{Writeback Operation:} Verify the \texttt{write\_en}, \texttt{write\_id}, and \texttt{write\_data} ports correctly update the internal state of the register file (can be checked by a subsequent read to the same \texttt{write\_id}).

    \item \textbf{Simultaneous Read/Write (Read-Old-Value):} Verify that in a single clock cycle, when a write operation occurs to address X (\texttt{write\_id=X}) and the \texttt{instruction} simultaneously reads from address X (\texttt{rs1=X} or \texttt{rs2=X}), the \texttt{read\_data1}/\texttt{read\_data2} outputs reflect the \textbf{old} value (verifying standard read-before-write register file behavior).

    \item Verify that when an R, I, S, or B-Type instruction has \texttt{instruction.rs1 >= 32}, \texttt{reg\_illegal} is asserted, subsequently asserting \texttt{instruction\_illegal}.

    \item Verify that when an R-Type instruction (e.g., \texttt{ADD x0, x1, x2}) has \texttt{rd=0}, \texttt{reg\_illegal} is asserted.
\end{itemize}

\subsection{Execute Stage Verification Items}

\textbf{Forwarding Logic:}
\begin{itemize}
    \item Verify correct selection from \texttt{data1}, \texttt{data2}, \texttt{mem\_forward\_data}, and \texttt{wb\_forward\_data}.
    \item Check data hazards resolution in cases FORWARD\_FROM\_EX and FORWARD\_FROM\_MEM.
\end{itemize}
\textbf{Control and Branch:}
\begin{itemize}
    \item Verify \texttt{pc\_src} asserted only for valid branch.
    \item Verify \texttt{jalr\_flag} and \texttt{jalr\_target\_offset} for jump instructions.
\end{itemize}
\textbf{ALU Integration:}
\begin{itemize}
    \item Cross-check ALU output consistency with Execute Stage’s \texttt{alu\_data}.
    \item Verify propagation of overflow and zero\_flag signals.
\end{itemize}

\section*{Grade3: End-to-End Verification Flow}
\subsection*{Build \& Simulation Orchestration}
\begin{itemize}
    \item \textbf{Script:} \texttt{exe\_v1/compile\_questa.do} deletes \texttt{work}, compiles DUT (\texttt{common.sv}, \texttt{alu.sv}, \texttt{execute\_stage.sv}), agents (\texttt{execute\_in}, \texttt{execute\_out}), env/test packages, and TB top with consistent \texttt{+incdir}. Runs \texttt{vsim} on \texttt{execute\_top\_tb} with \texttt{+UVM\_TESTNAME=execute\_top\_test}, then dumps coverage (\texttt{coverage\_report.txt}) and triages warnings/errors into \texttt{logs/}.
    \item \textbf{Topology:} The TB top binds VIFs to DUT, sets them in the config DB, and instantiates the UVM test which creates the environment.
\end{itemize}

\subsection*{Environment Composition}
\begin{itemize}
    \item \textbf{Agents:} \texttt{execute\_in\_agent} drives and monitors operand/control/forwarding inputs. \texttt{execute\_out\_agent} monitors DUT outputs. Each has a coverage subscriber for stimulus/response bins.
    \item \textbf{Reference Model:} \texttt{execute\_stage\_ref\_model} consumes input transactions from a TLM FIFO, recomputes expected execute-stage outputs (ALU op, forwarding selection, branch/JALR handling, overflow) and publishes \texttt{execute\_out\_tx} on \texttt{ref\_ap}.
    \item \textbf{Scoreboard:} \texttt{execute\_stage\_scoreboard} has two get/peek ports connected via FIFOs to expected (ref model) and actual (out monitor). It compares \texttt{control\_out}, \texttt{alu\_data}, \texttt{memory\_data}, \texttt{pc\_src}, \texttt{jalr\_flag}, \texttt{jalr\_target\_offset}, \texttt{pc\_out}, and \texttt{overflow}, counting pass/fail and reporting in \texttt{report\_phase}.
    \item \textbf{Sequences:} \texttt{execute\_top\_default\_seq} in \texttt{run\_phase} randomizes transactions and starts them on the \texttt{execute\_in} sequencer; the driver asserts \texttt{valid} for one beat per item.
\end{itemize}

\subsection*{Reference Model Detailing}
\begin{itemize}
    \item \textbf{Forwarding Resolution:} Selects operands per \texttt{forward\_rs1/rs2} (NONE/EX/MEM), overriding \texttt{data1/data2} with \texttt{mem\_forward\_data}/\texttt{wb\_forward\_data} where requested.
    \item \textbf{ALU/Jumps:} Uses \texttt{alu\_calc} to model all ALU ops including branches (zero flag inversion for B\_BNE/B\_BLT/B\_BGE/B\_LTU/B\_GEU). For J/JALR encodings it forces \texttt{alu\_data=pc+4} and sets \texttt{jalr\_flag}/\texttt{jalr\_target\_offset} when appropriate.
    \item \textbf{Store Payload:} \texttt{memory\_data} is either forwarded store data (for S-type) or the original \texttt{data2}. \texttt{pc\_src} reflects branch decision; \texttt{pc\_out} passes through.
    \item \textbf{Overflow:} Sets overflow only for ADD/SUB based on signed overflow detection; zero flag is derived from the computed result.
\end{itemize}

\subsection*{Coverage Instrumentation}
\begin{itemize}
    \item \textbf{execute\_in\_coverage:} Coverpoints on operand sign bits, PC alignment, encoding type, ALU op, alu\_src, branch flag, mem read/write dir, mem size/sign, reg\_write, forwarding selectors. Crosses encoding with forwarding to ensure hazard scenarios are hit.
    \item \textbf{Collection Points:} Monitors publish transactions to coverage subscribers; coverage is enabled via agent configs (\texttt{coverage\_enable}).
    \item \textbf{Closure:} UCDB produced by \texttt{coverage report -code bcesf -assert -cvg}; target is functional bins hit across encoding types, forwarding combos, branch outcomes, mem size/sign directions.
\end{itemize}

\subsection*{Stimulus and Checking Flow}
\begin{itemize}
    \item \textbf{Drive:} Default virtual sequence randomizes \texttt{execute\_tx} fields (operands, immediates, control encoding/op, forwarding selectors).
    \item \textbf{Sample:} \texttt{execute\_in\_monitor} captures inputs each cycle; \texttt{execute\_out\_monitor} captures DUT outputs. Both feed coverage and scoreboard/reference paths.
    \item \textbf{Compare:} Scoreboard waits for matching expected/actual, then compares field-by-field; mismatches raise \texttt{UVM\_ERROR} with expected vs actual details.
\end{itemize}

\section*{Grade4: Decode+Execute Cascaded Verification Flow}
\subsection*{Build \& Simulation Orchestration}
\begin{itemize}
    \item \textbf{Script:} \texttt{generated\_joint\_tb/sim/compile\_questa.do} compiles DUTs for the cascaded flow: \texttt{decode\_stage} from \texttt{generated\_execute\_tb/dut} and \texttt{execute\_stage}/\texttt{alu} from \texttt{exe\_v1/dut}. It also compiles decode agents from \texttt{generated\_decode\_tb} and the \texttt{execute\_out} agent from \texttt{exe\_v1}, then builds the joint env/test/top.
    \item \textbf{Run:} Simulates \texttt{joint\_top\_tb} with \texttt{+UVM\_TESTNAME=joint\_top\_test}; dumps UCDB and log triage into \texttt{generated\_joint\_tb/sim/logs/}.
\end{itemize}

\subsection*{Environment Composition}
\begin{itemize}
    \item \textbf{Agents:} \texttt{decode\_in\_agent} and \texttt{decode\_wb\_agent} are active drivers; \texttt{decode\_out\_agent} and \texttt{execute\_out\_agent} are passive monitors for the cascaded DUT outputs.
    \item \textbf{Env:} \texttt{joint\_top\_env} instantiates \texttt{joint\_scoreboard}, \texttt{joint\_ref\_model}, and \texttt{joint\_coverage} and connects decode/execute analysis streams to each checker.
\end{itemize}

\subsection*{Reference Model and Scoreboard}
\begin{itemize}
    \item \textbf{Decode Checks:} \texttt{joint\_scoreboard} caches \texttt{decode\_in} instruction/PC, recomputes expected \texttt{control\_signals} and illegal flags, and verifies \texttt{read\_data1/2} against an internal regfile model (gated when decode fails or instruction is illegal). It also flags illegal writes to \texttt{x0}.
    \item \textbf{Execute Checks:} \texttt{joint\_ref\_model} converts \texttt{decode\_out} transactions into expected \texttt{execute\_out} (ALU result, \texttt{pc\_src}, \texttt{memory\_data}) using \texttt{seq\_id} for alignment. The scoreboard compares \texttt{control\_out}, \texttt{alu\_data}, \texttt{memory\_data}, and \texttt{pc\_src} per instruction.
\end{itemize}

\subsection*{Coverage Instrumentation}
\begin{itemize}
    \item \textbf{cg\_joint:} Covers encoding, ALU op, mem controls, reg\_write, branch decision, and illegal flag, with cross coverage on \texttt{pc\_src} and control combinations.
    \item \textbf{cg\_instr:} Per-instruction bins for R/I/S/B/U/J opcodes to ensure decode-to-execute propagation across all instruction classes.
    \item \textbf{cg\_wb:} Tracks \texttt{x0} write attempts via \texttt{decode\_wb} monitoring.
\end{itemize}

\subsection*{Stimulus and Checking Flow}
\begin{itemize}
    \item \textbf{Drive:} \texttt{joint\_top\_test} runs \texttt{decode\_in\_default\_seq} and \texttt{decode\_wb\_default\_seq} to generate instruction streams and writeback updates.
    \item \textbf{Sample:} Decode/execute monitors publish transactions each cycle; \texttt{pc\_out} and \texttt{seq\_id} are used to align decode outputs with execute outputs.
    \item \textbf{Compare:} Scoreboard compares decode correctness and execute correctness in one flow, validating the cascade from decode to execute.
\end{itemize}


% \section{Bug Detection Strategy}

% \begin{itemize}
%     \item Use \textbf{scoreboard} to compare DUT outputs with a SystemVerilog reference model.
%     \item Add \textbf{assertions (SVA)}:
%     \begin{itemize}
%         \item \texttt{assert property (result == 0 -> zero\_flag == 1);}
%         \item \texttt{assert property (control inside \{ADD,SUB\} |-> overflow == expected);}
%         \item \texttt{assert property (forwarding path matches selected signal);}
%     \end{itemize}
%     \item Use directed tests for corner cases (e.g., overflow, max/min shift).
% \end{itemize}

\section{Coverage Plan}

\subsection{Functional Coverage}

\textbf{Joint (Decode+Execute) coverage in \texttt{generated\_joint\_tb} will cross:}
\begin{itemize}
    \item \texttt{encoding} $\times$ \texttt{pc\_src}, \texttt{alu\_op} $\times$ \texttt{pc\_src}.  
    \textit{Ensures branch/jump decisions align with decoded control and ALU op in the cascaded flow.}

    \item \texttt{mem\_read}, \texttt{mem\_write}, \texttt{reg\_write} $\times$ \texttt{instruction\_illegal}.  
    \textit{Ensures control combinations are covered while excluding illegal-control cases.}

    \item \texttt{mem\_size} $\times$ \texttt{mem\_sign}.  
    \textit{Covers byte/half/word and sign/zero extension behaviors through decode-to-execute.}

    \item Per-instruction bins across R/I/S/B/U/J opcodes (cg\_instr).  
    \textit{Ensures each instruction class is observed through the cascaded scoreboard/coverage.}

    \item \texttt{x0} write attempts via decode\_wb stream (cg\_wb).  
    \textit{Ensures x0 protection is exercised in joint verification.}
\end{itemize}

\textbf{Execute Stage coverage in \texttt{exe\_v1} will cross:}
\begin{itemize}
    \item Operand forwarding paths $\times$ control type.  
    \textit{Ensures all instruction types are tested under every forwarding condition.}

    \item Branch decisions (\texttt{pc\_src}) $\times$ zero\_flag.  
    \textit{Confirms branch logic behaves correctly for both taken and not-taken cases.}

    \item Instruction encoding types (\texttt{encoding\_t}).  
    \textit{Verifies that all instruction formats (R/I/S/B/U/J) are executed in simulation.}

    \item Immediate data extension cases.  
    \textit{Ensures both sign-extension and zero-extension immediates are covered.}

    \item Control signal forwarding (\texttt{control\_in} $\times$ \texttt{control\_out}).  
    \textit{Confirms control signals are properly propagated between pipeline stages.}

    \item JALR branch target computation (\texttt{jalr\_flag} $\times$ \texttt{jalr\_target\_offset}).  
    \textit{Validates that JALR-type branch targets are calculated correctly.}
\end{itemize}


\section{Verification Closure Criteria}

\begin{itemize}
    \item All functional coverage points reached ($\geq 95\%$).
    \item Code coverage $\geq 90\%$.
    \item All assertions pass without failure.
    \item No mismatches reported by scoreboard.
    \item All directed corner cases verified.
\end{itemize}

\section{Deliverables}

\begin{itemize}
    \item UVM environment source files.
    \item Verification plan and coverage report.
    \item Simulation waveforms for representative cases.
    \item Summary report of detected bugs and fixes.
\end{itemize}

\end{document}
